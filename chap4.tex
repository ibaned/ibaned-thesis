%%%%%%%%%%%%%%%%%%%%%%%%%%%%%%%%%%%%%%%%%%%%%%%%%%%%%%%%%%%%%%%%%%%
%                                                                 %
%                            CHAPTER FOUR                         %
%                                                                 %
%%%%%%%%%%%%%%%%%%%%%%%%%%%%%%%%%%%%%%%%%%%%%%%%%%%%%%%%%%%%%%%%%%%

\chapter{SCALABLE PARALLEL MESH ADAPTATION}

\section{Definition}

What does it mean to scale well.

\begin{equation} \label{eq:big-o-scale}
t = O\left(\frac{N}{P}\lg(P)\right)
\end{equation}

A list of requirements for scalability:

\begin{enumerate}
\item Never collect data of size $O(P)$ on one processor
\end{enumerate}

\section{Scalable Non-Blocking Collectives}

Half of the PCU paper goes here; the key algorithms
that got PHASTA to really large core counts and
that PETSc/Trilinos are still catching up on for
matrix assembly.

\section{Entity-Level Communication}

Something which both PCU and Omega\_h do is simulate
mesh entities sending messages to each other, while
doing this much more efficiently than the naive approach.

\section{Remotes and Owners}

Explain remote copies and owners, and how they become
part of the two structures from \ref{chap:struct}.

\section{Migration}

Both the PUMI/APF and Omega\_h migration algorithms
are quite different from what has been published for FMDB,
and they are essentially what it means for a mesh
data structure to be parallel.

\section{Ghosting}

As with migration, describe what is done in both PUMI/APF
and Omega\_h.

\section{Parallel Cavity Operations}

\subsection{Dynamic Migration}

The PUMI/APF ``Cavity Operator" system that supports MeshAdapt.

\subsection{Independent Sets}

The Omega\_h independent set system that supports adaptation
with help from ghosting.

\section{On-Node Parallelism}

Second half of the PCU paper describes how we were able to use
threading without rewriting everything.

Omega\_h content on how to do threading and GPUs properly
when the code is rewritten.

\section{Determinism}

Discuss things done by Omega\_h to achieve
determinism on heterogeneous architectures.

\subsection{Adjacency Inversion}

The key algorithms Ben and I came up with go here.

\subsection{Order-Independent Sums}

%%% Local Variables:
%%% mode: latex
%%% TeX-master: t
%%% End:

