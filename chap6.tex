%%%%%%%%%%%%%%%%%%%%%%%%%%%%%%%%%%%%%%%%%%%%%%%%%%%%%%%%%%%%%%%%%%%
%                                                                 %
%                            CHAPTER SIX                          %
%                                                                 %
%%%%%%%%%%%%%%%%%%%%%%%%%%%%%%%%%%%%%%%%%%%%%%%%%%%%%%%%%%%%%%%%%%%

\chapter{CONCLUSIONS AND FUTURE WORK}

\section{Conclusions}

\section{Future Work}

\subsection{Convergence of PUMI and Omega\_h}
\label{sec:converge}

Section \ref{sec:omega_h-struct} presented the rationale
for the development of Omega\_h as a separate effort from PUMI,
which opens the question of whether it is possible
to combine the two codes in the future.
A combination in this case would be defined as a strict
union of their capabilities.
In order to retain the ability to execute efficiently
on GPUs and other shared memory hardware, a parallel
\texttt{for} loop programming model such as that introduced
in Section \ref{sec:openmp} would need to be used throughout.
The MDS structure design which allows single additions
and removals (see Section \ref{sec:sisc_soa}) would likely
need to be replaced with a static design and independent
set algorithms as is done in Omega\_h.
The resulting structure would need to be augmented to
accept more than just simplices, as described in Section \ref{sec:sisc_mstruct}.
Finally, several MeshAdapt algorithms not discussed
in this thesis would need to be re-designed to use parallel
\texttt{for} loops.
