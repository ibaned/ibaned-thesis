%%%%%%%%%%%%%%%%%%%%%%%%%%%%%%%%%%%%%%%%%%%%%%%%%%%%%%%%%%%%%%%%%%%
%                                                                 %
%                            CHAPTER SIX                          %
%                                                                 %
%%%%%%%%%%%%%%%%%%%%%%%%%%%%%%%%%%%%%%%%%%%%%%%%%%%%%%%%%%%%%%%%%%%

\chapter{CONCLUSIONS AND FUTURE WORK}

\section{Conclusions}

\section{Future Work}

\subsection{Convergence of PUMI and Omega\_h}

Section \ref{sec:two_codes} presented the rationale
for the development of Omega\_h as a separate effort from PUMI,
which opens the question of whether it is possible
to combine the two codes in the future.
% A combination is simply defined as a code which
% has a union of their capabilities, both theoretical
% and practical.

% If a code is to make any use of current and near-future GPUs,
% then it must be written in the more restrictive programming
% model that Omega\_h operates in.
% This applies equally well to any proposed combination of the
% two codes.
% This in turn requires a redesign of algorithms
% and structures that currently violate those restrictions,
% which includes the majority of the PUMI code.
