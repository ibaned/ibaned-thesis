%%%%%%%%%%%%%%%%%%%%%%%%%%%%%%%%%%%%%%%%%%%%%%%%%%%%%%%%%%%%%%%%%%%
%                                                                 %
%                            APPENDICES                           %
%                                                                 %
%%%%%%%%%%%%%%%%%%%%%%%%%%%%%%%%%%%%%%%%%%%%%%%%%%%%%%%%%%%%%%%%%%%

\appendix    % This command is used only once!
%\addcontentsline{toc}{chapter}{APPENDICES}             %toc entry  or:
\addtocontents{toc}{\parindent0pt\vskip12pt APPENDICES} %toc entry, no page #

\chapter{Theoretical Aspects}

\section{Upward Adjacency Bounds}

\subsection{Tetrahedra Around a Vertex}

If geometric aspects are not taken into account, there
is limit on the number of simplices that may share a vertex.
For example, one can mesh a disk by separating it into arbitrarily
thin slices, each of which becomes a triangle.
Therefore, our proof of upper bounds on simplices sharing a vertex
must be based on an assumed geometric restriction.
We will begin with the most straightforward restriction, that
of solid angles, and correlate it to the mean ratio quality measure
defined by Equation \ref{eq:tet_mean_ratio} from Section \ref{sec:def_quality}.

Each tetrahedron adjacent to a vertex forms a solid angle at
the corner where the adjacency occurs.
For a given vertex, the sum of these solid angles for all
adjacent tetrahedra cannot exceed $4\pi$, the solid angle of a sphere.
Conversely, if there are $n$ tetrahedra adjacent to one vertex,
then one or more of the tetrahedra will obey Inequality
\ref{eq:solid_angle_degree}, where $\mathbf{\Omega}$ is the
solid angle of the relevant corner.

We assume that the way to maximize the mean ratio of a tetrahedron
that has one solid angle equal to $\mathbf{\Omega}$ is to have
its other three corners form an equilateral triangle.

Since the mean ratio is scale-invariant, we can consider this
without loss of generality for a tetrahedron $(o,a,b,c)$ where
$o$ is the center of a unit sphere and $(a,b,c)$ are on the surface
of that sphere, and form an equilateral triangle
as show in Figure \ref{fig:solid_angle}.
Let $\theta$ be the angle $\angle oab = \angle obc = \angle oca$,
Intuitively, as $\theta$ increases from zero to $\frac23\pi$,
the solid angle $\mathbf{\Omega}$ at $o$ monotonically
increases from zero to $2\pi$.
The exact relation between $\mathbf{\Omega}$ and $\theta$ is given by Equation
\ref{eq:solid2side}
using an intermediate $\phi$ (the dihedral angle between any pair of triangular
faces meeting at $o$).

\begin{figure}
\begin{center}
\includegraphics[width=0.4\textwidth]{solid_angle.png}
\caption{Maximizing quality versus solid angle}
\label{fig:solid_angle}
\end{center}
\end{figure}

\begin{equation} \label{eq:solid_angle_degree}
\mathbf{\Omega} \leq \frac{4\pi}{n}
\end{equation}

\begin{gather} \label{eq:solid2side}
\phi = \arccos\left(\frac{\cos\theta - \cos^2\theta}{\sin^2\theta}\right) \\
\mathbf{\Omega} = 3\phi - \pi = 3\arccos\left(\frac{\cos\theta - \cos^2\theta}{\sin^2\theta}\right) - \pi
\end{gather}

\begin{equation} \label{eq:surf_tri_height}
h = \frac{\sqrt{3}}{2}l = \sqrt{3}\sin\left(\frac{\theta}{2}\right)
\end{equation}

\begin{equation} \label{eq:surf_tri_inradius}
r = \frac{h}{3} = \frac{1}{\sqrt{3}}\sin\left(\frac{\theta}{2}\right)
\end{equation}

\begin{equation} \label{eq:surf_tri_circumradius}
R = 2r = \frac{2}{\sqrt{3}} \sin\left(\frac{\theta}{2}\right)
\end{equation}

\begin{equation} \label{eq:surf_tri_area}
A = \frac{\sqrt{3}}{4}l^2
\end{equation}

\begin{equation} \label{eq:tet_height}
H = \sqrt{1-R^2}
\end{equation}

\begin{equation} \label{eq:tet_volume}
V = \frac13 A H
\end{equation}

\begin{equation} \label{eq:msl}
l_{\text{MS}} = \frac16(3l^2 + 3) = \frac12(l^2 + 1)
\end{equation}

\begin{equation} \label{angle_quality}
\mathcal{Q} = \left(\frac{V^2}{\gamma^2 l_{\text{MS}}^3}\right)^{\frac13}
\end{equation}
