%%%%%%%%%%%%%%%%%%%%%%%%%%%%%%%%%%%%%%%%%%%%%%%%%%%%%%%%%%%%%%%%%%%
%                                                                 %
%                            APPENDICES                           %
%                                                                 %
%%%%%%%%%%%%%%%%%%%%%%%%%%%%%%%%%%%%%%%%%%%%%%%%%%%%%%%%%%%%%%%%%%%

\appendix    % This command is used only once!
%\addcontentsline{toc}{chapter}{APPENDICES}             %toc entry  or:
\addtocontents{toc}{\parindent0pt\vskip12pt APPENDICES} %toc entry, no page #

\chapter{Theoretical Aspects}

\section{Upward Adjacency Bounds}

If geometric aspects are not taken into account, there
is limit on the number of simplices that may share a vertex.
For example, one can mesh a disk by separating it into arbitrarily
thin slices, each of which becomes a triangle.
Therefore, our proof of upper bounds on simplices sharing a vertex
must be based on an assumed geometric restriction.
We will begin with the most straightforward restriction, that
of solid angles, and correlate it to the mean ratio quality measure
defined by Equation \ref{eq:tet_mean_ratio} from Section \ref{sec:def_quality}.

\begin{equation}
\phi = \arccos\left(\frac{\cos\theta - \cos^2\theta}{\sin^2\theta}\right)
\end{equation}

\begin{equation}
\mathbf{\Omega} = 3\phi - \pi
\end{equation}

\begin{equation} \label{eq:solid_angle_degree}
n = \left\lfloor \frac{4\pi}{\mathbf{\Omega}} \right\rfloor
\end{equation}

\begin{equation} \label{eq:surf_tri_height}
h = \frac{\sqrt{3}}{2}l = \sqrt{3}\sin\left(\frac{\theta}{2}\right)
\end{equation}

\begin{equation} \label{eq:surf_tri_inradius}
r = \frac{h}{3} = \frac{1}{\sqrt{3}}\sin\left(\frac{\theta}{2}\right)
\end{equation}

\begin{equation} \label{eq:surf_tri_circumradius}
R = 2r = \frac{2}{\sqrt{3}} \sin\left(\frac{\theta}{2}\right)
\end{equation}

\begin{equation} \label{eq:surf_tri_area}
A = \frac{\sqrt{3}}{4}l^2
\end{equation}

\begin{equation} \label{eq:tet_height}
H = \sqrt{1-R^2}
\end{equation}

\begin{equation} \label{eq:tet_volume}
V = \frac13 A H
\end{equation}

\begin{equation} \label{eq:msl}
l_{\text{MS}} = \frac16(3l^2 + 3) = \frac12(l^2 + 1)
\end{equation}

\begin{equation} \label{angle_quality}
\mathcal{Q} = \left(\frac{V^2}{\gamma^2 l_{\text{MS}}^3}\right)^{\frac13}
\end{equation}
