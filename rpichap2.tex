%%%%%%%%%%%%%%%%%%%%%%%%%%%%%%%%%%%%%%%%%%%%%%%%%%%%%%%%%%%%%%%%%%%
%                                                                 %
%                            CHAPTER TWO                          %
%                                                                 %
%%%%%%%%%%%%%%%%%%%%%%%%%%%%%%%%%%%%%%%%%%%%%%%%%%%%%%%%%%%%%%%%%%%

\chapter{FUNDAMENTAL DEVELOPMENTS}

This chapter is like Chapter \ref{chap:back} but
fills in ``background" items which were developed in
this work and not well covered in the literature.
It is difficult to say where it should end, i.e.
it starts to encompass all the ``pieces" the code

\section{Conformal Simplex Meshes}

\subsection{Simplices}

\subsubsection{Interpretation of Mean Ratio}

\subsection{Representation}

\subsubsection{Adjacency Cache}

\subsubsection{Orientation Codes}

\section{Mesh Adaptation}

\subsection{Cavity Modification Operators}

\subsubsection{Edge Splits versus Templates}

This was previously explored by Jean-Fran\c{c}ois though...

\subsubsection{Multiple Layers of Sliver Removal}

\subsection{Metric Field}

\subsubsection{Indentity Metric}

\subsubsection{Prediction and Control of Element Counts}

\section{Programming Model}

\subsection{On-Device Preference}

Prefer to run everything on the coprocessor device
when there is one, because transferring data
is very expensive.

\subsection{Immutable Array Structures}

%%% Local Variables:
%%% mode: latex
%%% TeX-master: t
%%% End:
