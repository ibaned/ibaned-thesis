%%%%%%%%%%%%%%%%%%%%%%%%%%%%%%%%%%%%%%%%%%%%%%%%%%%%%%%%%%%%%%%%%%%
%                                                                 %
%                            CHAPTER TWO                          %
%                                                                 %
%%%%%%%%%%%%%%%%%%%%%%%%%%%%%%%%%%%%%%%%%%%%%%%%%%%%%%%%%%%%%%%%%%%

\chapter{ARRAY-BASED MESH REPRESENTATIONS}
\label{chap:struct}

\section{Rationale}

Various requirements for storing of entities, adjacencies,
ability to modify.
Indicate how each requirement affects complexity.

\section{Related Work}

Specifics of mesh representations in the literature.
This can probably come from the SISC paper, plus
some more SCOREC references.

\section{Two Paradigms, Two Structures}

A lead-in describing the two structures and their difference
in abilities.

\section{PUMI/APF Data Structure}

{\color{red} SISC attribution}

\begin{enumerate}
\item The representation centers around graph theoretic interpretations
of topological adjacency.
\item The mesh can remain topologically consistent with and associated with
geometric model entities.
\item The common element types of FE/FV methods can coexist in one structure.
\item Additional data can be associated with entities to implement
high order basis functions, including for geometric approximation.
\item A mesh can be modified by adding and removing single entities in constant time.
\item The entire mesh is stored in a few contiguous dynamic arrays.
\end{enumerate}

One key contribution of this paper is to show that the latter two properties,
array storage and rapid single-entity modification, are not mutually exclusive
and can be combined in a viable way.

\section{Omega\_h Data Structure}

\subsection{Adjacency Cache}

\subsection{Alignment Codes}

\section{Data Structure Performance}

Query times, modification times, etc. for both structures.

%%% Local Variables:
%%% mode: latex
%%% TeX-master: t
%%% End:
