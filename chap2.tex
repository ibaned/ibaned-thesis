%%%%%%%%%%%%%%%%%%%%%%%%%%%%%%%%%%%%%%%%%%%%%%%%%%%%%%%%%%%%%%%%%%%
%                                                                 %
%                            CHAPTER TWO                          %
%                                                                 %
%%%%%%%%%%%%%%%%%%%%%%%%%%%%%%%%%%%%%%%%%%%%%%%%%%%%%%%%%%%%%%%%%%%

\chapter{ARRAY-BASED MESH REPRESENTATIONS}
\label{chap:struct}

\section{Rationale}

Various requirements for storing of entities, adjacencies,
ability to modify.
Indicate how each requirement affects complexity.

\section{Related Work}

Specifics of mesh representations in the literature.
This can probably come from the SISC paper, plus
some more SCOREC references.

\section{Two Paradigms, Two Structures}

A lead-in describing the two structures and their difference
in abilities.

The end of this section can discuss possibilities of convergence.

\section{PUMI/APF Data Structure}

SISC paper on PUMI/APF structure goes here.

\section{Omega\_h Data Structure}

\subsection{Adjacency Cache}

\subsection{Alignment Codes}

\section{Data Structure Performance}

Query times, modification times, etc. for both structures.

%%% Local Variables:
%%% mode: latex
%%% TeX-master: t
%%% End:
