%%%%%%%%%%%%%%%%%%%%%%%%%%%%%%%%%%%%%%%%%%%%%%%%%%%%%%%%%%%%%%%%%%%
%                                                                 %
%                            ABSTRACT                             %
%                                                                 %
%%%%%%%%%%%%%%%%%%%%%%%%%%%%%%%%%%%%%%%%%%%%%%%%%%%%%%%%%%%%%%%%%%%

\specialhead{ABSTRACT}

Mesh adaptation is a technique which dynamically modifies the
mesh being used to approximately solve a Partial Differential
Equation (PDE) in order to improve aspects of the approximate solution
including the computer time and memory used to compute it
as well as its level of accuracy.
Even with the use of mesh adaptation, computing ever more accurate
PDE solutions requires significant computer time and memory,
motivating the use of supercomputers, which are constructed as
networks of cooperating computational hardware.
Trends in the computer hardware industry at large are introducing
heterogeneous designs for current leadership-class supercomputers,
which is both an opportunity and a challenge for programs
aiming to make use of these machines.

This thesis presents implementations of mesh adaptation
which are designed with memory efficient cache-friendly
data structures and algorithms which can effectively leverage
both distributed memory parallelism and shared memory
parallelism (including GPUs).
The data structures used in these implementations are widely
applicable to other tasks involving meshes, and the programming
paradigms introduced are general enough to be of use in most
programs targeting leadership-class supercomputers.
The implementations presented are being used by
several simulation codes in production, and are available
as open-source tools so they may continue providing value
to the scientific community.

Several improvements to the design of mesh adaptation programs
are presented, including solution transfer methods which
preserve mass and momentum, methods for the maintenance of
high-quality elements, scalable and deterministic methods
for hybrid parallelization of mesh modification operations,
and a combination of modification operators which reduce
implementation complexity without sacrificing effectiveness.
