%%%%%%%%%%%%%%%%%%%%%%%%%%%%%%%%%%%%%%%%%%%%%%%%%%%%%%%%%%%%%%%%%%%
%                                                                 %
%                            CHAPTER THREE                        %
%                                                                 %
%%%%%%%%%%%%%%%%%%%%%%%%%%%%%%%%%%%%%%%%%%%%%%%%%%%%%%%%%%%%%%%%%%%

\chapter{CAVITY-BASED CONFORMAL MESH ADAPTATION}

\section{Definition}

Introduce the basics of cavity modifications.
Compare/contrast with non-conformal non-cavity methods here.

\section{Related Work}

Specifics of how adaptation is done in the literature.
Focus is on parallel-agnostic aspects.

\section{Solution Transfer in a Cavity}

Emphasize numerical / performance advantage
over full-mesh methods.
Results can be included for each particular
method if it warrants one.

\subsection{Conserving Integral Quantities}

\section{Size Field Construction}

\subsection{Identity Size Field}

\subsection{Targeting an Element Count}

\section{MeshAdapt Methods}

Focuses on serial aspects, i.e. refinement templates.

\section{Omega\_h Methods}

Focuses on serial aspects, i.e. single splits and
quality restrictions.

\section{Serial Adaptation Performance}

Both absolute performance and perhaps comparing
different methods.

%%% Local Variables:
%%% mode: latex
%%% TeX-master: t
%%% End:


