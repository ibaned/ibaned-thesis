%%%%%%%%%%%%%%%%%%%%%%%%%%%%%%%%%%%%%%%%%%%%%%%%%%%%%%%%%%%%%%%%%%%
%                                                                 %
%                            CHAPTER ONE                          %
%                                                                 %
%%%%%%%%%%%%%%%%%%%%%%%%%%%%%%%%%%%%%%%%%%%%%%%%%%%%%%%%%%%%%%%%%%%

\chapter{INTRODUCTION AND BACKGROUND}
\label{chap:intro}

\section{Introduction}

\subsection{Goals}
\label{sec:intro_goals}

Parallel mesh adaptation for current and next generation supercomputers
that can support a wide variety of applications.

\subsection{Requirements}

A more precise version of Section \ref{sec:intro_goals},
itemizing specific features and citing references for them.

\subsection{Overview of Contributions}

Also indicates where in the thesis they are.
This is probably the best time to introduce the
different pieces of software: PUMI, APF, PCU, MeshAdapt, Omega\_h.

\section{Definitions}

\subsection{Nomenclature}

Probably a big table goes here, with explaining text.

\subsection{Conformal Meshes}

\subsection{Petascale and Heterogeneous Supercomputers}

Define things like processors, ``node", accelerator, etc.

\subsubsection{Programming Environments}

Define MPI, threads, CUDA terminology.

Latex requires at least one citation to
compile: \cite{edwards2013kokkos}.

\section{Related Works}

Just mentions works and their subject, don't go into detail here.

%%% Local Variables:
%%% mode: latex
%%% TeX-master: t
%%% End:
