%%%%%%%%%%%%%%%%%%%%%%%%%%%%%%%%%%%%%%%%%%%%%%%%%%%%%%%%%%%%%%%%%%%
%                                                                 %
%                            CHAPTER ONE                          %
%                                                                 %
%%%%%%%%%%%%%%%%%%%%%%%%%%%%%%%%%%%%%%%%%%%%%%%%%%%%%%%%%%%%%%%%%%%

\chapter{INTRODUCTION AND BACKGROUND}
\label{chap:intro}

\section{Introduction}

A wide variety of aerospace, mechanical, and nuclear engineering
problems require the solution of complex Partial Differential
Equations (PDEs) in time and space.
For efficiency and reliability, the solutions to these PDEs are
found by computers.
Computers are equipped with a limited amount of memory to
store information, and must use a mathematical representation
of a PDE solution that can be described using
a limited amount of information.
For many engineering problems of interest, the exact solution
as described by any known representation would require an
infinite amount of information, therefore approximate
solutions are sought.
A certain minimal amount of memory and processing power
are required to obtain the lowest accuracy approximate
solutions, and obtaining more accurate solutions requires
more memory and/or processing power.
For these reasons, computers with ever-increasing amounts
of memory and processing power are designed and built to
increase the accuracy of existing solutions and
to solve previously unsolvable engineering problems.
At any given point in history, the computers with the
most memory and processing power are called supercomputers.

\section{Definitions}

\subsection{Nomenclature}

{\color{red} Part of the nomenclature is attributed to SISC}

\begin{tabular}{l|l}
Processing Unit & \\
Topological Complex & A breakdown of a domain in Cartesian space into \\
topological entities \\
Mesh & A topological complex whose entities have simple shape \\
Entity & A topological entity of a mesh \\
Vertex & A 0-dimensional entity \\
Edge & A 1-dimensional entity \\
Face & A 2-dimensional entity \\
Region & A 3-dimensional entity \\
Element & An entity not bounding another entity \\
\end{tabular}

\subsection{Conformal Meshes}

\subsection{Petascale and Heterogeneous Supercomputers}

Define things like processors, ``node", accelerator, etc.

\subsubsection{Programming Environments}

Define MPI, threads, CUDA terminology.

Latex requires at least one citation to
compile: \cite{edwards2013kokkos}.

\section{Goals and Requirements}
\label{sec:intro_goals}

{\color{red} These goals are from the SISC paper in review, will
need to be attributed}

An unstructured mesh simulation code relies heavily on
multiple core capabilities to deal with the mesh,
and the range of features available at this level constrain
the capabilities of the simulation as a whole.
As such, the long-term goal towards which this thesis
contributes is the development of a mesh handling system
with the following capabilities:

\begin{enumerate}
\item The flexibility to adapt to evolving meshes
\item The ability to represent any of the conforming meshes typically
used by Finite Element (FE) and Finite Volume (FV) methods
\item Low memory use
\item High locality of storage
\item Highly scalable implementation for distributed memory computers
\item The ability to parallelize work inside heterogeneous
supercomputer nodes
\end{enumerate}

The first goal is the most consequential; supporting adaptivity
is the reason for much of the complexity in the structure
and its difference compared to many non-adaptive mesh structures
(see Section \ref{sec:adapt} for further discussion).
In particular, we present a derivation for a family of structures
with the following properties:

\section{Requirements}

A more precise version of Section \ref{sec:intro_goals},
itemizing specific features and citing references for them.

\subsection{Overview of Contributions}

Also indicates where in the thesis they are.
This is probably the best time to introduce the
different pieces of software: PUMI, APF, PCU, MeshAdapt, Omega\_h.

\section{Related Works}

Just mentions works and their subject, don't go into detail here.

%%% Local Variables:
%%% mode: latex
%%% TeX-master: t
%%% End:
